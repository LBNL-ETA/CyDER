%% Generated by Sphinx.
\def\sphinxdocclass{report}
\documentclass[letterpaper,10pt,english]{sphinxmanual}
\ifdefined\pdfpxdimen
   \let\sphinxpxdimen\pdfpxdimen\else\newdimen\sphinxpxdimen
\fi \sphinxpxdimen=.75bp\relax

\usepackage[utf8]{inputenc}
\ifdefined\DeclareUnicodeCharacter
 \ifdefined\DeclareUnicodeCharacterAsOptional\else
  \DeclareUnicodeCharacter{00A0}{\nobreakspace}
\fi\fi
\usepackage{cmap}
\usepackage[T1]{fontenc}
\usepackage{amsmath,amssymb,amstext}
\usepackage{babel}
\usepackage{times}
\usepackage[Bjarne]{fncychap}
\usepackage{longtable}
\usepackage{sphinx}

\usepackage{geometry}
\usepackage{multirow}
\usepackage{eqparbox}

% Include hyperref last.
\usepackage{hyperref}
% Fix anchor placement for figures with captions.
\usepackage{hypcap}% it must be loaded after hyperref.
% Set up styles of URL: it should be placed after hyperref.
\urlstyle{same}

\addto\captionsenglish{\renewcommand{\figurename}{Fig.}}
\addto\captionsenglish{\renewcommand{\tablename}{Table}}
\addto\captionsenglish{\renewcommand{\literalblockname}{Listing}}

\addto\extrasenglish{\def\pageautorefname{page}}

\setcounter{tocdepth}{2}



\title{PV Forecasting Module}
\date{May 02, 2017}
\release{1.0.0}
\author{LBNL - Building Technology and Urban Systems Division}
\newcommand{\sphinxlogo}{}
\renewcommand{\releasename}{Release}
\makeindex

\begin{document}

\maketitle
\sphinxtableofcontents
\phantomsection\label{\detokenize{index::doc}}



\chapter{Introduction}
\label{\detokenize{introduction:introduction}}\label{\detokenize{introduction::doc}}\label{\detokenize{introduction:user-guide}}\label{\detokenize{introduction:id1}}
This user guide explains how to install and use the CyDER forecasting module.
The forecasting module is a software package written in C which allows
users to predict future Photovoltaic generation values from historical data.
The algorithm uses a Nonlinear Autoregressive Exogenous neural network model.
This neural network is trained with historical PV data and potentially other
independent time-series such as weather or surrounding PV generation. Once
the network is trained it can then be used for prediction of future values.
We use the Fast Artifical Neural Network Library (\sphinxtitleref{FANN http://leenissen.dk/})
to develop this neural network model.


\chapter{Download}
\label{\detokenize{download:download}}\label{\detokenize{download::doc}}\label{\detokenize{download:id1}}
The forecast module is only released through source, there are no pre-built
binaries available yet. To install the module please follow the instructions at {\hyperref[\detokenize{build::doc}]{\sphinxcrossref{\DUrole{doc}{Build}}}}.


\section{Release 1.0.0 (TODO DATE)}
\label{\detokenize{download:release-1-0-0-todo-date}}
Download forecast.tar.gz.
\phantomsection\label{\detokenize{build:build}}

\chapter{Build}
\label{\detokenize{build:id1}}\label{\detokenize{build::doc}}\label{\detokenize{build:check-install}}

\section{Dependencies}
\label{\detokenize{build:dependencies}}
This project has two external dependencies:
- We use the \sphinxhref{https://github.com/HardySimpson/zlog}{zlog} library as our logging mechanism. This enables the module log important parameters and options
every time the tool is run which allows for easier debugging.
- In order to unit test our code we use the \sphinxhref{https://libcheck.github.io/check/}{check} library, this is optional so it doesn't need to be installed.


\section{Installing Dependencies}
\label{\detokenize{build:installing-dependencies}}

\subsection{Zlog Installation}
\label{\detokenize{build:zlog-installation}}
To install zlog please follow the instructions at \sphinxhref{https://github.com/HardySimpson/zlog}{zlog\_install}.


\subsection{Check Installation {[}Optional{]}}
\label{\detokenize{build:check-installation-optional}}
Installation of this library is optional, however, if you want to run the unit tests then
you must install it. Please follow the instructions at \sphinxhref{https://libcheck.github.io/check/web/install.html}{check\_install}.


\section{Installation}
\label{\detokenize{build:installation}}
First, make sure you've installed the dependencies outlined above and then follow these steps:
\begin{enumerate}
\item {} 
Clone the repository

\item {} 
Navigate to the root directory (./forecast) and create a build directory

\end{enumerate}

\begin{sphinxVerbatim}[commandchars=\\\{\}]
mkdir build
\end{sphinxVerbatim}
\begin{enumerate}
\setcounter{enumi}{2}
\item {} 
Navigate to the build directory

\end{enumerate}

\begin{sphinxVerbatim}[commandchars=\\\{\}]
\PYG{n+nb}{cd} build/
\end{sphinxVerbatim}
\begin{enumerate}
\setcounter{enumi}{3}
\item {} 
Run cmake from the build directory

\end{enumerate}

\begin{sphinxVerbatim}[commandchars=\\\{\}]
cmake ../
\end{sphinxVerbatim}

NOTE: If you installed check and want to build the unit tests along with the main executable then run the following:

\begin{sphinxVerbatim}[commandchars=\\\{\}]
cmake ../ \PYGZhy{}DFORECAST\PYGZus{}ENABLE\PYGZus{}TESTING\PYG{o}{=}ON
\end{sphinxVerbatim}
\begin{enumerate}
\setcounter{enumi}{4}
\item {} 
Compile and install the files

\end{enumerate}

\begin{sphinxVerbatim}[commandchars=\\\{\}]
make
make install
\end{sphinxVerbatim}


\chapter{Usage of the Forecasting Module}
\label{\detokenize{usage:usage}}\label{\detokenize{usage::doc}}\label{\detokenize{usage:usage-of-the-forecasting-module}}
Before embarking on this section please make sure you've followed the instructions
at {\hyperref[\detokenize{build::doc}]{\sphinxcrossref{\DUrole{doc}{Build}}}}.

At a high level, the forecasting module allows the user to do two things:
1. Train a neural network model
2. Run predictions on unseen data using the model trained

In this section, we will go over the options available to the user for each of these modes.
In general, many of the operations allowed by the module require a data file, please see
{\hyperref[\detokenize{help::doc}]{\sphinxcrossref{\DUrole{doc}{Help}}}} for information about the data file format.

A summary of these options can be found by executing the following command:

\begin{sphinxVerbatim}[commandchars=\\\{\}]
./forecast\PYGZhy{}main \PYGZhy{}\PYGZhy{}help
\end{sphinxVerbatim}


\chapter{Setting the Environment {[}Optional{]}}
\label{\detokenize{usage:setting-the-environment-optional}}
In order to log output to a file we need to set a few environment variables, this is
because our module needs to know where the files should be saved to and where the log
configurations are saved. To set these variables automatically execute the following commands:

\begin{sphinxVerbatim}[commandchars=\\\{\}]
\PYG{c+c1}{\PYGZsh{} Make sure you\PYGZsq{}re at root}
\PYG{n+nb}{cd} forecast/
\PYG{n+nb}{source} bin/setenv.sh
\end{sphinxVerbatim}

The setenv.sh script is a simple helper script that sets all the environment variables to
their default values. The log outputs will be saved inside the forecast/log folder by default.

Below is a table of the customizable environment variables:

\noindent\begin{tabulary}{\linewidth}{|L|L|}
\hline
\sphinxstylethead{\relax 
Environment Variable
\unskip}\relax &\sphinxstylethead{\relax 
Description
\unskip}\relax \\
\hline
FORECAST\_LOG\_CONFIGURATION
&
Points to the ZLOG configuration file
\\
\hline
FORECAST\_LOG\_FILE
&
Points to the file where the log output will be saved
\\
\hline
FORECAST\_LOG\_LEVEL
&
Specifies the verbosity of our logs
\\
\hline\end{tabulary}


The FORECAST\_LOG\_LEVEL can take the following values:
\begin{itemize}
\item {} 
DEBUG

\item {} 
INFO

\item {} 
NOTICE

\item {} 
WARN

\item {} 
ERROR

\item {} 
FATAL

\item {} 
NONE

\end{itemize}


\section{Training a Neural Network}
\label{\detokenize{usage:training-a-neural-network}}
To train a neural network, we need minimally a training data file that is only used
for the purposes of training. A network can be trained right off the box using the
following command:

\begin{sphinxVerbatim}[commandchars=\\\{\}]
./forecast\PYGZus{}main \PYGZhy{}\PYGZhy{}training\PYGZhy{}file\PYG{o}{=}\PYGZlt{}FILE\PYGZgt{}
\end{sphinxVerbatim}

This command will train a neural network with default parameters, this tends to yield
poor results in general. Training a good neural network requires the tuning of many
hyper-parameters to achieve good convergence. The table below lists every hyper-parameter
and the corresponding command-line option that the user can use.

\noindent\begin{tabulary}{\linewidth}{|L|L|L|}
\hline
\sphinxstylethead{\relax 
Hyper-Parameter
\unskip}\relax &\sphinxstylethead{\relax 
Command Line Option
\unskip}\relax &\sphinxstylethead{\relax 
Type
\unskip}\relax \\
\hline
Initial Learning Rate
&
--initial-learning-rate
&
Float
\\
\hline
Tapped Delay Lines
&
--num-tapped-delay-lines
&
Int
\\
\hline
Momentum
&
--initial-momentum
&
Float
\\
\hline
Hidden Layers
&
--hidden-layers
&
Comma Separated Int
\\
\hline
Number of Epochs
&
--max-epochs
&
Int
\\
\hline
Error Function
&
--error-functions
&
\sphinxquotedblleft{}RMSE\sphinxquotedblright{} or \sphinxquotedblleft{}MSE\sphinxquotedblright{}
\\
\hline
Desired Error
&
--desired-error
&
Float
\\
\hline
Learning Rate Adaptation
&
--learning-rate-adaptation
&
\sphinxquotedblleft{}LINEAR\sphinxquotedblright{} or \sphinxquotedblleft{}BOLD\sphinxquotedblright{}
\\
\hline\end{tabulary}


Another good option to use is --test-file. This option allows you to pass test data
on which to evaluate your model. Now we know how to modify hyper-parameters and we can
pass a training and testing data file. As an example, suppose we want to train a neural
network with two hidden layers of size 10, momentum 0.9, learning rate of 0.01 and 3
tapped delay lines. To do so, we would run the following command:

\begin{sphinxVerbatim}[commandchars=\\\{\}]
./forecast\PYGZhy{}main \PYGZhy{}\PYGZhy{}training\PYGZhy{}file\PYG{o}{=}TRAIN\PYGZus{}FILE \PYGZhy{}\PYGZhy{}test\PYGZhy{}file\PYG{o}{=}TEST\PYGZus{}FILE
                \PYGZhy{}\PYGZhy{}hidden\PYGZhy{}layers\PYG{o}{=}\PYG{l+m}{10},10 \PYGZhy{}\PYGZhy{}initial\PYGZhy{}momentum\PYG{o}{=}\PYG{l+m}{0}.9 \PYGZhy{}\PYGZhy{}initial\PYGZhy{}learning\PYGZhy{}rate\PYGZhy{}0.01
                \PYGZhy{}\PYGZhy{}num\PYGZhy{}tapped\PYGZhy{}delay\PYGZhy{}lines\PYG{o}{=}\PYG{l+m}{3}
\end{sphinxVerbatim}

The output to this command is two serialized neural network files called fann.net and narx.net.
When running predictions we will load our models from these files so make sure to keep them
around.


\section{Running Predictions}
\label{\detokenize{usage:running-predictions}}
Running predictions is very similar to training a neural network, there are two inputs
needed: the serialized neural network and a prediction data file. If you haven't yet, please look
at {\hyperref[\detokenize{help::doc}]{\sphinxcrossref{\DUrole{doc}{Help}}}} to see the prediction file format as it differs slightly from the training and
testing data format.

Given these inputs, run the following command:

\begin{sphinxVerbatim}[commandchars=\\\{\}]
./forecast\PYGZhy{}main \PYGZhy{}\PYGZhy{}narx\PYGZhy{}file\PYG{o}{=}NARXNET \PYGZhy{}\PYGZhy{}fann\PYGZhy{}file\PYG{o}{=}FANNNET \PYGZhy{}\PYGZhy{}predict\PYGZhy{}file\PYG{o}{=}PREDICT\PYGZus{}FILE
\end{sphinxVerbatim}

The output of the command is a CSV file with the prediction for each timestep. Nothing else
is required in order to run predictions .


\chapter{Help}
\label{\detokenize{help::doc}}\label{\detokenize{help:help}}\label{\detokenize{help:id1}}

\section{Data File Format}
\label{\detokenize{help:data-file-format}}
For many of the operations within the forecasting module, a data file is required.
This section outlines how the module expects the file to be formated.


\subsection{Training And Testing File}
\label{\detokenize{help:training-and-testing-file}}
The training and testing files contain at the first line three comma
numbers: number\_of\_entries, number\_of\_inputs, number\_of\_outputs. This is followed
by an interleaved lines of comma seperated inputs and comma separated outputs.

An example is shown below:

\begin{sphinxVerbatim}[commandchars=\\\{\}]
NUM\PYGZus{}ENTRIES, NUM\PYGZus{}INPUTS, NUM\PYGZus{}OUTPUTS
input\PYGZus{}1, input\PYGZus{}2 \PYGZsh{} Entry 1
\PYG{g+gh}{output\PYGZus{}1, output\PYGZus{}2, output\PYGZus{}3 \PYGZsh{} Entry 1}
\PYG{g+gh}{...}
\PYG{c+cp}{...}
input\PYGZus{}1, input\PYGZus{}2 \PYGZsh{} Entry NUM\PYGZus{}ENTRIES
output\PYGZus{}1, output\PYGZus{}2, output\PYGZus{}3 \PYGZsh{} Entry NUM\PYGZus{}ENTRIES
\end{sphinxVerbatim}

A concrete example is shown below:

\begin{sphinxVerbatim}[commandchars=\\\{\}]
1000 2 3
1.000000 0.000000
1.000000 0.000000 1.000000
0.999950 0.010000
0.999900 0.009999 0.999900
0.999800 0.019999
0.999600 0.019995 0.999600
0.999550 0.029996
0.999100 0.029982 0.999100
\end{sphinxVerbatim}


\subsection{Prediction File}
\label{\detokenize{help:prediction-file}}
This type of file is identical to the training \& testing files but in this case
num\_outputs must be set to zero and only lines representing the input are used.
During prediction, we don't the real outputs available so it isn't necessary to pass
these to the module.

A concrete example is shown below:

\begin{sphinxVerbatim}[commandchars=\\\{\}]
300 2 0
\PYGZhy{}0.171866 0.985120
\PYGZhy{}0.181708 0.983352
\PYGZhy{}0.191533 0.981486
\PYGZhy{}0.201338 0.979522
\PYGZhy{}0.211123 0.977460
\PYGZhy{}0.220887 0.975300
\PYGZhy{}0.230628 0.973042
\end{sphinxVerbatim}


\chapter{Acknowledgments}
\label{\detokenize{acknowledgments::doc}}\label{\detokenize{acknowledgments:acknowledgments}}
The development of this documentation was supported
by the Assistant Secretary for Energy Efficiency and Renewable Energy,
Office of Building Technologies of the U.S. Department of Energy,
under contract No. xxx.

The following people contributed to the development of this module:
\begin{itemize}
\item {} 
Rafael Alberto Rivera Soto, Lawrence Livermore National Laboratory

\item {} 
Brian Kelley, Lawrence Livermore National Laboratory

\end{itemize}


\chapter{Disclaimer}
\label{\detokenize{disclaimer::doc}}\label{\detokenize{disclaimer:disclaimer}}
This document was prepared as an account of work sponsored by the United States
Government. While this document is believed to contain correct information, neither the
United States Government nor any agency thereof, nor The Regents of the University of
California, nor any of their employees, makes any warranty, express or implied, or assumes
any legal responsibility for the accuracy, completeness, or usefulness of any information,
apparatus, product, or process disclosed, or represents that its use would not infringe
privately owned rights. Reference herein to any specific commercial product, process, or
service by its trade name, trademark, manufacturer, or otherwise, does not necessarily
constitute or imply its endorsement, recommendation, or favoring by the United States
Government or any agency thereof, or The Regents of the University of California. The
views and opinions of authors expressed herein do not necessarily state or reflect those of the
United States Government or any agency thereof or The Regents of the University of
California.


\chapter{Copyright and License}
\label{\detokenize{legal:copyright-and-license}}\label{\detokenize{legal::doc}}

\section{Copyright}
\label{\detokenize{legal:copyright}}
\sphinxquotedblleft{}CyDER v01\sphinxquotedblright{} Copyright (c) 2017, The Regents of the University of California,
through Lawrence Berkeley National Laboratory (subject to receipt of any
required approvals from the U.S. Dept. of Energy).  All rights reserved.

If you have questions about your rights to use or distribute this software,
please contact Berkeley Lab's Innovation \& Partnerships Office at  \sphinxhref{mailto:IPO@lbl.gov}{IPO@lbl.gov}.

NOTICE.  This Software was developed under funding from the U.S.
Department of Energy and the U.S. Government consequently retains certain rights.
As such, the U.S. Government has been granted for itself and others acting on
its behalf a paid-up, nonexclusive, irrevocable, worldwide license in the Software
to reproduce, distribute copies to the public, prepare derivative works,
and perform publicly and display publicly, and to permit others to do so.


\section{License Agreement}
\label{\detokenize{legal:license-agreement}}
\sphinxquotedblleft{}CyDER v01\sphinxquotedblright{} Copyright (c) 2017, The Regents of the University of California,
through Lawrence Berkeley National Laboratory (subject to receipt of any
required approvals from the U.S. Dept. of Energy).  All rights reserved.

Redistribution and use in source and binary forms, with or without modification,

are permitted provided that the following conditions are met:

(1) Redistributions of source code must retain the above copyright notice,
this list of conditions and the following disclaimer.

(2) Redistributions in binary form must reproduce the above copyright notice,
this list of conditions and the following disclaimer in the documentation
and/or other materials provided with the distribution.

(3) Neither the name of the University of California,
Lawrence Berkeley National Laboratory, U.S. Dept. of Energy nor the names
of its contributors may be used to endorse or promote products derived
from this software without specific prior written permission.

THIS SOFTWARE IS PROVIDED BY THE COPYRIGHT HOLDERS AND CONTRIBUTORS
\sphinxquotedblleft{}AS IS\sphinxquotedblright{} AND ANY EXPRESS OR IMPLIED WARRANTIES, INCLUDING, BUT NOT LIMITED
TO, THE IMPLIED WARRANTIES OF MERCHANTABILITY AND FITNESS FOR A PARTICULAR
PURPOSE ARE DISCLAIMED. IN NO EVENT SHALL THE COPYRIGHT OWNER OR CONTRIBUTORS
BE LIABLE FOR ANY DIRECT, INDIRECT, INCIDENTAL, SPECIAL, EXEMPLARY,
OR CONSEQUENTIAL DAMAGES (INCLUDING, BUT NOT LIMITED TO, PROCUREMENT
OF SUBSTITUTE GOODS OR SERVICES; LOSS OF USE, DATA, OR PROFITS;
OR BUSINESS INTERRUPTION) HOWEVER CAUSED AND ON ANY THEORY OF LIABILITY,
WHETHER IN CONTRACT, STRICT LIABILITY, OR TORT (INCLUDING NEGLIGENCE OR OTHERWISE)
ARISING IN ANY WAY OUT OF THE USE OF THIS SOFTWARE, EVEN IF ADVISED OF THE POSSIBILITY OF SUCH DAMAGE.

You are under no obligation whatsoever to provide any bug fixes, patches,
or upgrades to the features, functionality or performance of the source code
(\sphinxquotedblleft{}Enhancements\sphinxquotedblright{}) to anyone; however, if you choose to make your Enhancements
available either publicly, or directly to Lawrence Berkeley National Laboratory,
without imposing a separate written license agreement for such Enhancements,
then you hereby grant the following license: a non-exclusive, royalty-free,
perpetual license to install, use, modify, prepare derivative works, incorporate
into other computer software, distribute, and sublicense such enhancements or
derivative works thereof, in binary and source code form.



\renewcommand{\indexname}{Index}
\printindex
\end{document}